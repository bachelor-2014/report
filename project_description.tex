\documentclass[a4paper,11pt]{article}
\usepackage[utf8]{inputenc}
\usepackage{graphicx}
\usepackage[english]{babel}
%\usepackage[vmargin=2.5cm]{geometry}
\usepackage[linktocpage=true]{hyperref}
\usepackage{pgfgantt}

\setcounter{secnumdepth}{0}

\hypersetup{
    colorlinks,
    citecolor=black,
    filecolor=black,
    linkcolor=black,
    urlcolor=black
}

\renewcommand{\labelenumii}{\theenumii}
\renewcommand{\theenumii}{\theenumi.\arabic{enumii}.}

\begin{document}

\section{Project description: `Stand-alone Evolutionary Robotic Platform'}

\subsection{Background and motivation}
This bachelor project resides within the scope of the EVOBLISS project
that seeks to develop a robotic platform for supporting research on
artificial, technological evolution with the goal of evolving microbial
fuels cells in terms of robustness, longevity, or adaptability in order
to improve wastewater cleanup. The robotic platform is based on an open
source 3D printer with extended functionality for handling liquids and
reaction vessels. Focus is on real time reaction based on feedback from
a series of sensors. The main motivation for the EVOBLISS project is
enhancing the understanding of living technologies and to gain an
insight in the design of bio-hybrid systems.

This bachelor project is based on the existing liquid handling robot as
described above. This robot is similar to a 3D printer, but with the
printer head being replaced with a syringe for handling liquid.
Furthermore, it is extended with a gripper for manipulating petri
dishes. The hardware is controlled by an Arduino unit, and the robotic
platform as a whole is controlled using software installed on a personal
computer connected to the Arduino.

The first part of the bachelor project focuses on modifying the setup by
removing the Arduino and the personal computer, replacing them with a
BeagleBoard. The BeagleBoard is to interact directly with the hardware
of the robotic platform while also fulfilling the role of the personal
computer running the software, providing a high level user interface for
interacting with the robot. The motivation for this is the hope that
making the robot a stand alone unit will remove sources of error when
using the robot, making it easier to use for people with no IT
background; users will not need to install any software on their own
computers, making it more accessible, and making it less likely that the
heterogeneity of computers introduce usage difficulties.

The second part of this bachelor project focuses on the use of a camera
as as sensor through the use of image analysis and manipulation. Having
a stationary camera as part of the robot has proven to not be
sufficient, as the area to be scanned often exceeds the area that can be
covered in a single image. We will therefore attempt to create a setup,
where the syringe is replaced with a camera, and where large areas can
be scanned by taking multiple images with the same camera and stitching
them together. This also covers investigating the difficulties
introduced by having a moving camera such as the motion blur introduced
by both the camera movement and the vibrations in the robot caused by
the motors. This second part is motivated by an actual case where such a
platform is needed, but where the sensor used is not necessarily an
ordinary camera but a similar device. The hope is therefore that a to
some extend generic solution of the setup can be created, where the
camera can be replaced by another similar device such as a microscope or
an OCT scanner head for wider applicability.

\subsection{Scope of the project}
The project consists of two parts: (1) Replacing the existing Arduino
unit and software with a BeagleBoard, and (2) creating a setup with a
movable camera for scanning large areas through use of image stitching.
The following elaborates on these parts:

\begin{enumerate}

    \item Replacing the existing Arduino unit and software with a BeagleBoard
    \begin{enumerate}
        \item
          Setup the operating system and application environment on the
          BeagleBoard. The board will be running a Linux distribution.
        \item
          Find/write drivers for the different hardware of the platform such as
          motors, camera, and other sensors. For the stepper motors, hardware
          drivers will be used for simplifying this task. Existing drivers will
          be used as far as this is possible.
        \item
          Interface with hardware from within the software running on the
          BeagleBoard. This includes low-level control of the stepper motors and
          the retrieval of data for the different sensors.
        \item
          Create an easy-to-use API on top of the hardware interfaces. This
          serves to simplify further interaction with the hardware. It will be
          based on our own estimations of requirements for such an API rather
          than actual requirements specification due to time limitations.
    \end{enumerate}

    \item Creating a setup with a movable camera for scanning large
    areas through use of image stitching
    \begin{enumerate}
        \item
          Create the actual hardware setup. It must support moving the camera
          along at least two axes.
        \item
          Implement image stitching. This will make use of existing
          implementations as far as possible.
        \item
          Experiment with moving the camera in order to find the method giving
          the best results. Examples of different methods are taking the images
          while moving the camera (without stopping) and taking the images while
          holding the camera still, moving it between taking the images
          (stopping between images).
        \item
          Create and easy-to-use API for interaction with the camera based
          scanner. As with the API on top of the hardware interfaces, this will
          be based on our own requirements estimations.
    \end{enumerate}

\end{enumerate}

\subsection{Time plan}
\begin{ganttchart}[vgrid, hgrid]{20}{20}
    \gantttitle{2014}{20} \\
    \gantttitlelist{"Jan","Feb","Mar","Apr","May"}{4} \\
    \gantttitlelist{2,...,21}{1} \\


    \ganttmilestone{Project base}{4} \ganttnewline
    \ganttmilestone{Submit for approval}{6} \ganttnewline
    \ganttmilestone{Handin}{20} \ganttnewline

    \ganttgroup{1. BeagleBoard}{1}{13} \\
    \ganttbar{1.1 OS and OpenCV}{1}{2} \\
    \ganttbar{1.2 Drivers}{1}{2} \\
    \ganttbar{1.3 Hardware interaction}{1}{2} \\
    \ganttbar{1.4 Hardware API}{1}{2} \\

    \ganttgroup{2. Camera}{4}{6} \\
    \ganttbar{2.1 Hardware setup}{4}{5} \\
    \ganttbar{2.2 Image stitching}{5}{6} \\
    \ganttbar{2.3 Camera experiments}{5}{6} \\
    \ganttbar{2.4 Camera API}{5}{6} \\

    \ganttgroup{Report}{1}{2}
\end{ganttchart}


\end{document}
